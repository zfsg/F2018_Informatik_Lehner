\documentclass[10pt,a4paper]{scrartcl}

\usepackage[english]{babel}
\usepackage{glossaries}\usepackage[xindy]{imakeidx}

%Set to 0 for making without examples
%Set to 1 for making with examples
\gdef\conditionmacro{1}

%\newenvironment{example}{
%\ifnum\conditionmacro=1}
%{\fi}

\input{../Headerfiles/Packages}
\input{../Headerfiles/Titles}
\input{../Headerfiles/Commands}
\input{../Headerfiles/ENVIRONMENTS}
\graphicspath{{Pictures/}}
\parindent 0pt

\title{Informatics}
\author{GianAndrea Müller}

\newtheorem{define}{Definition}

\makenoidxglossaries
\newglossaryentry{Algorithm}{name={Algorithm},description={An algorithm is a set of rules that defines a sequence of operations to get to the solution of a problem.}}
\newglossaryentry{Language}{name={Language},description={A programming language is a set of instructions for a computer that can be used to write programs that implement algorithms.}}
\newglossaryentry{Syntax}{name={Syntax},description={The syntax of a computer language is the set of rules that defines the combinations of symbols that are considered to be a correctly structured document or fragment in that language. Is a program grammatically correct?}}
\newglossaryentry{Semantics}{name={Semantics},description={The semantics of a computer language define how the language has to be interpreted. What is the meaning of a certain program?}}
\newglossaryentry{Editor}{name={Editor},description={A program that allows writing code. There exist powerful editors that can check syntactical correctness on the fly.}}
\newglossaryentry{Compiler}{name={Compiler},description={A compiler translates a program written in a programming language to machine code, such that it can be executed by the machine.}}
\newglossaryentry{Computer}{name={Computer},description={A computer is a device that is capable of executing machine code.}}
\newglossaryentry{Comments}{name={Comments},description={Comments document the implemented algorithm within the program for the reader. They are ignored by the compiler.}}
\newglossaryentry{IncDir}{name={Include Directives},description={Include directives specify the additional libraries used in a program.}}
\newglossaryentry{Main}{name={The main function},description={The main function exists in every cpp-program. It is unique and contains the all instructions necessary to execute the program.}}
\newglossaryentry{Statement}{name={Statement},description={Statements are the building blocks of a program. They are executed sequentially and end with a semicolon.}}
\newglossaryentry{Declaration}{name={Declaration},description={A declaration introduces a new name to the program.}}
\newglossaryentry{Definition}{name={Definition},description={A definition introduces a body to a name within the program.}}
\newglossaryentry{Initialization}{name={Initialization},description={An initialization introduces a value to a defined name and body.}}
\newglossaryentry{Literals}{name={Literals},description={Literals represent constant value within the program. They have a defined type and value.}}
\newglossaryentry{Variables}{name={Variables},description={Variables represent possibly changing values within the program. They have name, type, value and address.}}
\newglossaryentry{Objects}{name={Objects},description={Objects represent values in the computer memory. They have type, adress and value. They can be named, but can also be anonymous. Described less generally an object can be a variable, a data structure, a function, or a method.}}
\newglossaryentry{Expressions}{name={Expressions},description={Expressions represent calculations. They are a combination of values, literals, operators and functions. They are primary if they consist of a single name/literal. Otherwise they are compound. They have type and value.}}
\newglossaryentry{Lvalue}{name={Lvalue},description={An lvalue is a changeable expression that has an address.}}
\newglossaryentry{Rvalue}{name={Rvalue},description={An rvalue is an expression that is not an lvalue. An rvalue cannot be changed. Every lvalue can be used as an rvalue but not vice-versa.}}
\newglossaryentry{Operator}{name={Operator},description={An operator connects expressions to compound expressions. It specifies the expected operand in type and if it is an rvalue or an lvalue. Operators have an arity.}}
\newglossaryentry{Arity}{name={Arity},description={Arity is the number of arguments or operands an operator or a function takes. For example there exist unary and binary operators.}}

\begin{document}
\begin{multicols*}{4}
\maketitle
\tableofcontents
\end{multicols*}

\begin{multicols*}{2}

%Nice to know
\newcommand{\ntn}[1]{\textcolor{blue}{#1}}

\ifnum\conditionmacro=1
\section{How to...}

\subsection{\ntn{... use this summary}}

This summary is an overview of the functionality of \verb.C++. in connection with the informatics course for mechanical engineers. It covers the content of the lectures but also contains additional information.

To emphasize the connection to the lecture all chapters containing purely additional information are marked in blue.

\subsection{\ntn{... correct compilation errors}}

Read error messages, review basic syntax, look for the additional semicolon.

\subsection{\ntn{... correct runtime errors}}

Use a \easyurl{debugger}{https://www.visualstudio.com/de/thank-you-downloading-visual-studio/?sku=Community&rel=15}.

\subsection{... approach problems}

\begin{enumerate}
\item Define your problem.
\item Find your algorithm.
\item Code feature.
\item Compile.
\item \verb+goto+ 3.
\end{enumerate}

\subsection{... find more information}

\begin{itemize}
\item \easyurl{Comprehensive Tutorial}{https://www.tutorialspoint.com/cplusplus/index.htm}
\item \easyurl{User friendly documentation}{http://www.cplusplus.com}
\item \easyurl{Extensive technical documentation}{http://en.cppreference.com/w/}
\end{itemize}
\fi

%Add all entries to glossary even though unused
\glsaddall

\printnoidxglossary[sort=def,title={Terms},nonumberlist=true,nopostdot=true]

\section{Positional Notation}

\subsection{Binary numbers}

\subsubsection{Floating point numbers}

\subsection{Hexadecimal numbers}

\section{Syntax}

\subsection{Basic program}

\begin{TPCpp}
#include <iostream>
//#include "local_header_file.h"

/*
 * Function declarations (and definitions)
 */

int main(int argc, char ** argv)
{
	  /*
	   * Function calls
	   */
    std::cout << "Hello World!" << std::endl;
    return 0;
}

/*
 * Function definitions
 */
\end{TPCpp}

\subsection{Identifiers}

A valid identifier, i.e. the name of a variable is:

\begin{itemize}
\item an arbitrarily long sequence of digits, underscores, lowercase and uppercase Latin letters.
\item not starting with a digit.
\item not starting with two or more underscores.
\item not starting with an underscore followed by a capital letter.
\item not a \easyurl{keyword}{http://en.cppreference.com/w/cpp/keyword} of cpp.
\end{itemize}

\ifnum\conditionmacro=1
More information on \easyurl{Identifiers}{http://en.cppreference.com/w/cpp/language/identifiers}.
\fi

\subsection{Comments}

C++ allows masking code such that it is not interpreted as part of the program. This enables documenting the program. There are different possibilities:

\begin{TPCpp}
// normal comment

/*
multi
line
comment
*/
\end{TPCpp}

Both versions can be nested:

\begin{TPCpp}
///*comment in a comment*/

/*
cout<<"Hello World!<<endl; //comment in a comment
*/
\end{TPCpp}

\subsection{Data Types}

\subsubsection{Primitive Types}

\ifnum\conditionmacro=1
\begin{TTable}[1]
{ll}
Type&Keyword\\\midrule
Boolean&bool\\
Character&char\\
Integer&int\\
Floating point&float\\
Double floating point&double\\
Valueless&void\\
\end{TTable}
\fi

\subsubsection{Type modifiers}

\ifnum\conditionmacro=1
There exist a number of type modifiers:

\begin{TTable}[1]
{ll}
Modifier&Effect\\\midrule
signed&variable interpreted as signed\\
unsigned&variable interpreted as unsigned\\
short&half number of allocated bits if possible\\
long&double number of allocated bits if possible\\
\end{TTable}

Based on the primitive types and their modifiers the spectrum of available types can be established. Their sizes differ depending on compiler and environment.
\fi

\begin{TTable}[1]
{llc}
Modifier&Typical Bit Width&Typical Range\\\midrule
char&1byte&-127 to 127\\
unsigned char&1byte&0 to 255\\
signed char&1byte&-127 to 127\\
int&4byte&-2'147'483'648 to 2'147'483'647\\
unsigned int&4bytes&0 to 4'294'967'295\\
signed int&4bytes&-2'147'483'648 to 2'147'483'647\\
short int&2bytes&-32'768 to 32'767\\
unsigned short int&2bytes&0 to 65'535\\
signed short int&2bytes&-32'768 to 32'767\\
long int&4bytes&-2'147'483'648 to 2'147'483'647\\
signed long int&4bytes&-2'147'483'648 to 2'147'483'647\\
unsigned long int&4bytes&0 to 4'294'967'295\\
float&4bytes&+/- 3.4e +/- 38 (~7 digits)\\
double&8bytes&+/- 1.7e +/- 308 (~15 digits)\\
long double&8bytes&+/- 1.7e +/- 308 (~15 digits)\\
\end{TTable}

\ifnum\conditionmacro=1
\subsubsection{\ntn{Find type sizes on your system}}
\begin{TPCpp}
#include <iostream>
using namespace std;

int main() {
   cout<< "Size: char : "<<sizeof(char)<<endl;
   cout<< "Size: int : "<<sizeof(int)<<endl;
   cout<< "Size: short int : "<<sizeof(short int)<<endl;
   cout<< "Size: long int : "<<sizeof(long int)<<endl;
   cout<< "Size: float : "<<sizeof(float)<<endl;
   cout<< "Size: double : "<<sizeof(double)<<endl;
   
   return 0;
}
\end{TPCpp}
\fi

\subsubsection{Type qualifiers}
%const volatile restrict

\subsubsection{Storage classes}
%auto register static extern mutable

\subsubsection{Lvalues and Rvalues}

\subsubsection{Variable Definition}

\subsubsection{union}

\subsubsection{enum}

\section{Variable Scope}

\subsection{Local Variables}

\subsection{Global Variables}

\section{Literals}

\subsection{Integer Literals}

\subsection{Floating-point Literals}

\subsection{Boolean Literals}

\subsection{Character Literals}

\subsection{String Literals}

\subsection{Defining constants}

\subsubsection{\#define}

\subsubsection{const}

\section{Operators}

\subsection{Arithmetic Operators}

\subsection{Relational Operators}

\subsection{Logical Operators}

\subsection{Bitwise Operators}

\subsection{Assignment Operators}

\subsection{Misc Operators}

\subsection{Operator Precedence and Associativity}

\begin{TTable*}{|c|lp{0.5\linewidth}|l|}{\small}
P.&Operator&Description&Associativity\\\hline
1&\verb+::+&Scope resolution&\multirow{6}{*}{Left-to-right}\\\cline{1-3}
\multirow{5}{*}{2}&\verb.a++ a--.&Suffix/postfix increment and decrement&\\
&\verb.type() type{}.&Functional cast&\\
&\verb.a().&Function call&\\
&\verb.a[].&Subscript&\\
&\verb+. ->+&Member access&\\\hline
\multirow{9}{*}{3}&\verb.++a --a.&Prefix increment and decrement&\multirow{9}{*}{Right-to-left}\\
&\verb.+a -a.&Unary plus and minus&\\
&\verb.! ~.&Locigal NOT and bitwise NOT&\\
&\verb.(type).&C-style cast&\\
&\verb.*a.&Dereference&\\
&\verb.&a.&Adress-of&\\
&\verb.sizeof.&Size-of&\\
&\verb.new new[].&Dynamic memory allocation&\\
&\verb.delete delete[].&Dynamic memory deallocation&\\\hline
4&\verb+.* ->*+&Pointer-to-member&\multirow{11}{*}{Left-to-right}\\
5&\verb+a*b a/b a%b+&Multiplication, division, remainder&\\\cline{1-3}
6&\verb.a+b a-b.&Addition and subtraction&\\\cline{1-3}
7&\verb.<< >>.&Bitwise left shift and right shift&\\\cline{1-3}
8&\verb.<=>.&Three-way comparison operator&\\\cline{1-3}
9&\verb.< <= > >=.&Relational operators&\\\cline{1-3}
10&\verb.== !=.&Relational operators&\\\cline{1-3}
11&\verb.&.&Bitwise AND&\\\cline{1-3}
12&\verb.^.&Bitwise XOR&\\\cline{1-3}
13&\verb.|.&Bitwise OR&\\\cline{1-3}
14&\verb.&&.&Logical AND&\\\cline{1-3}
15&\verb.||.&Logical OR&\\\hline
\multirow{7}{*}{16}&\verb.a?b:c.&Ternary conditional&\multirow{7}{*}{Right-to-left}\\
&\verb.throw.&throw operator&\\
&\verb.=.&Direct assignment&\\
&\verb.+= -=.&\multirow{4}{*}{Compound assignments}&\\
&\verb.*= /= %=.&&\\
&\verb.<<= >>=.&&\\
&\verb.&= ^= |=.&&\\\hline
17&\verb.,.&Comma&Left-to-right\\
\end{TTable*}

\ifnum\conditionmacro=1
\subsubsection{How to use this table}

\begin{TPCpp}
cout<<a&&b;    //(cout<<a)&&b;

*p++           //*(p++);

a = b = c = d; //a = (b =(c = d)));

a + b - c;     //(a + b) - c;

delete ++*p;   //delete(++(*p))
\end{TPCpp}

\begin{enumerate}
\item By its precedence \verb+<<+ is evaluated before \verb+&&+.
\item By its precedence \verb.++. is evaluated before \verb.*..
\item Operators with the same precedence are evaluated based on their associativity. For right-to-left associative operators as \verb+=+, the evaluation proceeds from right to left.

Thus the assignments made in line 5 are in the order of their execution: \verb+c = d;+ which returns a reference to \verb+c+, \verb+b = c;+ which returns a reference to \verb+b+ and \verb+a = b;+.
\item Operators with the same precedence are evaluated based on their associativity. For left-to-right associative operators as \verb-+- and \verb+-+ the evaluation proceeds from left to right.
\item \verb.++()., \verb.*(). and \verb+delete+ have the same precedence, and are thus evaluated based on their associativity, which is right-to-left. Therefore \verb.++(). is evaluated after \verb.*(). and \verb+delete+ is evaluated last.
\end{enumerate}
\fi


\section{Loop Types}

\subsection{while}

\subsection{for}

\subsection{do...while}

\subsection{Loop Control Statements}
%break continue goto

\section{Conditional Statements}

\subsection{if}

\subsection{if...else}

\subsection{switch}

\subsection{? : Operator}

\section{Functions}

\subsection{Structure}

\subsection{Declaration and Definition}

\subsection{Calling a function}

\subsection{Function Arguments}

\subsubsection{Call by Value}

\subsubsection{Call by Reference}
%and by pointer - both in same category?

\subsubsection{Default Values for Parameters}

\subsection{Recursion}

\subsection{Inline Functions}

\section{Arrays}

\section{vector}

\section{Strings}

\subsection{C-Style Character String}

\subsection{string}

\section{Pointers}

\section{References}

\section{Input/Output}

\section{struct}

\section{class}

\subsection{Class Members}

\subsection{Class Access Modifiers}

\subsection{Constructor and Destructor}

\subsection{Copy Constructor}

\subsection{friend}

\subsection{this}

\subsection{Static Members}

\section{Inheritance}

\subsection{Access Control and Inheritance}

\section{Overloading}

\subsection{Function overloading}

\subsection{Operator overloading}

\subsubsection{Overloadable operators}

\section{Polymorphism}

\section{Dynamic Memory}

\section{Namespaces}

\section{Templates}

\section{Preprocessor}

\section{Signal Handling}

\section{Standard Template Library}

\section{Libraries}

\subsection{iostream}

\subsection{math}

\subsection{ctime}







\end{multicols*}
\end{document}
